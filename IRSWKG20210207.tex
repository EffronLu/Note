\documentclass[geye,green,normal,cn,founder]{elegantnote}
\title{IRSWKG笔记}
\author{Effron Lu}
\institute{SEU}
\version{1.00}
\date{\zhdate{2021/02/07}}      
\begin{document}
\maketitle
\section{信息协调(Inforamtion Reconciliation)}
信息协调实现方法分为两种:
\begin{enumerate}
    \item[*] 基于Cascade信息协调协。Cascade协议需要Alice和Bob有一个交互的过程,在交互中纠错。
    \item[*] 基于纠错码(Error Correcting Code, ECC),基于纠错码的信息协调只需要一方发送信息即可纠错,但相比较于Cascade协议而 言泄露的信息量会大一点,计算开销也更加大。 
\end{enumerate}

基于纠错编码的信息协调实现流程如下:

\begin{enumerate}
    \item Alice随机选择一个比特串$r$,然后通过纠错编码的方式对$r$进行编码,记作$Enc(r)$, $Enc(r)$的长度和量化后的比特串$q_A$的长度一样。 
    \item Alice计算$SS = q_A \otimes Enc(r)$。 
    \item Alice把$SS$发送给Bob。 
    \item Bob收到$SS$后和自己量化得到的比特$q_B$进行异或
    \item 使用纠错算法得到 $\tilde{r}$,如果$q_B$和$q_A$不一样的比特数在纠错算法的纠错能力范围之内,那么$\tilde{r}$和$r$一样。
    \item 最后Bob使用纠错码的编码算法对$\tilde{r}$进行编码后和$SS$进行一个异或,就可以得到比特串$q_A$了
    \item 上述过程归纳为:$q_B^{\prime} = SS \otimes Enc(Dec(SS \otimes q_B)) = SS \otimes Enc(\tilde{r})$
\end{enumerate}

主流的纠错编码算法:BCH编码、Turbo编码、卷积编码、 LDPC编码、RS编码.
\section{隐私放大(Privacy Amplification)}
隐私放大的实现方法可以分为:
\begin{enumerate}
    \item 模糊提取器(Fuzzy Extractor)
    \item 全域哈希函数来(Universal Hashing Functions)
    \item \textbf{密码学哈希函数实现}(Cryptographic Hash Function)
\end{enumerate}

密码学哈希函数实现流程:量化出来的这些比特,每256比特为一个分组,隐私增强后输出的最终密钥⻓度减短为128比特。
  
\end{document}
